\documentclass[conference]{IEEEtran}
\IEEEoverridecommandlockouts

\usepackage{cite}
\usepackage{amsmath,amssymb,amsfonts}
\usepackage{algorithmic}
\usepackage{graphicx}
\usepackage{xspace}
\usepackage{multirow}
\usepackage{multicol}


\def\BibTeX{{\rm B\kern-.05em{\sc i\kern-.025em b}\kern-.08em
	T\kern-.1667em\lower.7ex\hbox{E}\kern-.125emX}}

\begin{document}
\title{CECS 524 Hello \LaTeX \\ {\small Welcome to the \LaTeX world}}

\author{\IEEEauthorblockN{Taina Gariglio Dias}
	\IEEEauthorblockA{\textit{Computer engineering and Computer Science} \\
		\textit{California State University Long Beach}\\
		Long Beach, USA\\
		jucheol.moon@csulb.edu}}

\maketitle

\begin{abstract}
This document is a model and instructions for \LaTeX. This and the IEEEtran.cls file define the components of your paper [title, text, heads, etc.].
\end{abstract}

\begin{IEEEkeywords}
component, formatting, style, styling, insert
\end{IEEEkeywords}

\section{Introduction}
This document is a model and instructions for \LaTeX.
\section{Ease of Use}

\subsection{Maintaining the Integrity of the Specifications}

The IEEEtran class file is used to format your paper and style the text. All margins, column widths, line spaces, and text fonts are prescribed; please do not alter them. Please do not revise any of the current designations.

\section{Styling Guide}

\subsection{Abbreviations and Acronyms}
Define Abbreviations and acronyms the first time they are used in the text, even after they have been defined in the abstract.
\subsection{Lists}
Bullet style list.
\begin{itemize}
    \item item 1
    \item item 2
    \item item 3
\end{itemize}
Number style list.
\begin{enumerate}
    \item item 1
    \item item 2
    \item item 3
\end{enumerate}

\subsection{Equations}
\begin{equation}
    \sum_{n=0}^{\infty} \dfrac{af^{n}}{n!}(x-a)^{n} 
\label{taylor}
\end{equation}
\eqref{taylor} is the famous Taylor series. Use ``(1)", not ``Eq. (1)" or ``equation (1)", except at the beginning of a sentence: ``Equation (1) is . . ."

    Taylor series in a text would be  $\sum_{n=0}^{\infty} \dfrac{af^{n}}{n!}(x-a)^{n}$.

\subsection{Algorithms}
\begin{algorithmic}
\STATE $i \gets 10$
\IF {$i\geq 5$}
    \STATE $i \gets i-1$
\ELSE 
    \IF {$i \leq 3$}
        \STATE $i \gets i+2$
    \ENDIF
\ENDIF

\end{algorithmic}
\subsection{Source codes}
\begin{verbatim}
public class HelloWorld {
    public static void main (String[] args) {
        System.out.println("Hello, World");
    }
}
\end{verbatim}

\subsection{Figures and Tables}
\paragraph{Positioning Figures and Tables} Figure captions should be 
below the figures; table heads should appear above the tables. Insert 
figures and tables after they are cited in the text. Use the abbreviation 
``Fig.~\ref{fig1}''.

\begin{table}[h]
    \label{tb1}
    \caption{Table Type Styles}
    \centering
    \begin{tabular}{|c||c|c|c|}
    \hline
    \textbf{Table}& \multicolumn{3}{|c|}{\textbf{Table Column Head}}\\
    \cline{2-4}
    \textbf{Head}& \textbf{\textit{Table column subhead}} &\textbf{\textit{Subhead}} &\textit{\textbf{Subhead}} \\
    \hline
    & & & \\
    \hline
    \end{tabular}
\end{table}

\begin{figure}[h]
    \centering
    \includegraphics[width=0.2\textwidth]{fig1.png}
    \caption{Working example}
    \label{fig1}
\end{figure}


\subsection{References}
Please number citations consecutively within brackets \cite{b1}. The sentence punctuation follows the bracket\cite{b2}. Refer simply to the reference number, as in\cite{b3}---do not use ``Ref. [3]" or ``reference [3]" except at the beginning of a sentence. 

\begin{thebibliography}{00}
\bibitem{b1} G. Eason, B. Noble, and I. N. Sneddon, ``On certain integrals of Lipschitz-Hankel type involving products of Bessel functions,'' Phil. Trans. Roy. Soc. London, vol. A247, pp. 529--551, April 1955.
\bibitem{b2} J. Clerk Maxwell, A Treatise on Electricity and Magnetism, 3rd ed., vol. 2. Oxford: Clarendon, 1892, pp.68--73.
\bibitem{b3} I. S. Jacobs and C. P. Bean, ``Fine particles, thin films and exchange anisotropy,'' in Magnetism, vol. III, G. T. Rado and H. Suhl, Eds. New York: Academic, 1963, pp. 271--350
\end{thebibliography}

\end{document}
